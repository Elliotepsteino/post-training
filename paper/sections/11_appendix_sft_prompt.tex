\section{SFT Filtering Prompt}
\label{sec:sft-prompt}
\begin{verbatim}
You label the minimum calendar year (between 2001 and 2025) required to answer
a question without temporal leakage. The label must never precede any fact
mentioned in the sample; when uncertain, err toward the later year so that no
future knowledge sneaks into earlier buckets.

You receive a dataset-specific question plus an answer bundle (which may contain
multiple sections).
These are supervised instruction-tuning pairs: treat the question as the user
prompt and the response as the assistant answer.
Pick the smallest year Y in [2001, 2025] so that a model with knowledge through
year Y could answer confidently, considering EVERYTHING in both the question and
the answer bundle. If no specific time-dependent knowledge is required, output
2001.

Rules:
- Identify all time-anchored entities in the question and answer bundle.
- For each entity, provide a best_estimate year plus a 95% confidence interval.
- Use the entity's founding/release/announcement year (not the future target year).
- Set overall "year" to the maximum upper bound across all entity confidence intervals.
- If a range is mentioned (e.g., "released between 2008 and 2015"), use that as
  the entity's interval.
- If information is older than 2001, still respond with 2001.
- Do not hallucinate years; use only dates grounded in the text or well-known facts.
- Additionally, assign the question to one category from this list:
  general_knowledge, math, coding, science, history, law, finance, health,
  creative_writing, multi_lingual, instruction_following, reasoning, other.

You may reason internally, but the final output must be a single JSON object only.
Do not include any extra text or code fences.

Return JSON with these required fields and meanings:
- "year": integer in [2001, 2025] for the minimum safe year.
- "confidence": "low" | "medium" | "high".
- "category": one of the allowed categories listed above.
- "justification": short reason for the chosen year.
- "entities": object mapping entity names to an object with:
  - "best_estimate": best estimate year for founding/release/announcement.
  - "confidence_interval_95": [yearA, yearB] containing the best estimate; yearA/yearB
    can be the same.
  - "search_query": a standalone query to verify the year estimate.
- If no entities are found, use an empty object for "entities".

Illustrative example (output only):
{"year": 2006, "confidence": "high", "category": "general_knowledge",
"justification": "Answer references tweets, a concept only available after
Twitter launched in 2006, so 2006 is the earliest safe year.",
"entities": {"tweet": {"best_estimate": 2006, "confidence_interval_95": [2006, 2006],
"search_query": "When was Twitter launched?"}}}

<question>
{sample.question}
</question>
<answer_bundle>
{sample.answer}
</answer_bundle>
Return JSON exactly in this schema:
{"year": 2001, "confidence": "low|medium|high",
"category": "one of the allowed categories",
"justification": "why year is required",
"entities": {"entityA": {"best_estimate": 2008, "confidence_interval_95": [2007, 2009],
"search_query": "When was entityA released?"},
"entityB": {"best_estimate": 2013, "confidence_interval_95": [2013, 2013],
"search_query": "When was entityB announced?"}}}
\end{verbatim}

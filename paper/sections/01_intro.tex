\section{Introduction}
\paragraph{Problem setting.}
We first fix notation. Let $x$ be a text sample and let $t \in \mathbb{R}$ denote time.
For each text sample $x$, our goal is to produce a probability distribution
\[
x \;\mapsto\; p_x(t),
\]
where $p_x(t)$ represents a probability density over times at which $x$ could have been written using only publicly available knowledge as of time $t$.
We refer to this notion of time as the \emph{epistemic availability time}.

We refer to the task of estimating $p_x(t)$ as \emph{temporal grounding}.
In this work, $p_x(t)$ is defined to reflect a downstream use case of interest: labeling and bucketing training data for a language model with a custom knowledge cutoff year $\tau$ (e.g., backtesting NLP-based models and what-if scenario analysis under historical knowledge constraints).
Operationally, we use LLM-based estimators to approximate $p_x(t)$ (or a year-discretized analogue), which can then be used to produce either a hard year label or a soft inclusion weight for training at cutoff $\tau$.

\paragraph{Desired properties of $p_x(t)$.}
For our downstream application, we highlight four desired properties that $p_x(t)$ should satisfy.
\begin{enumerate}
    \item \textbf{No temporal leakage.}
    A model trained with data up to year $\tau$ should not encode knowledge of facts that became publicly knowable after $\tau$.
    Accordingly, $p_x(t)$ should assign negligible mass to times earlier than $\tau$ for text that depends on post-$\tau$ knowledge.

    \item \textbf{Timely availability of explicit events.}
    If a text references an explicit, time-anchored event that became publicly knowable in year $y$ (e.g., a product launch), then $p_x(t)$ should place substantial mass at or after year $y$, ensuring that a model trained through year $y$ can correctly state the fact.

    \item \textbf{Historically appropriate implicit knowledge.}
    Text often reflects implicit social knowledge or public salience rather than discrete events.
    For example, the statement ``There has been a lot of talk about global warming on the news'' could plausibly be written across a wide range of years, with increasing plausibility over time.
    A model trained only through the 1970s should not exhibit the level or framing of concern characteristic of much later discourse; such behavior would indicate temporal leakage.
    Accordingly, $p_x(t)$ should capture gradual shifts in public understanding rather than sharp thresholds.

    \item \textbf{Alignment with epistemic availability rather than authorship time.}
    Explicit cues often imply a single earliest admissible year, while implicit cues naturally induce a distribution over a range of years.
    These distinctions correspond to different notions of time, including \emph{event time}, \emph{epistemic availability time}, \emph{authorship or publication time}, and \emph{stylistic or discourse time}.
    For our purposes, epistemic availability time is the most appropriate notion: an event may be known to insiders before becoming public, but a time-bounded model should not rely on insider knowledge.
    Likewise, authorship time can be misleading, since a contemporary author may deliberately write text consistent with much earlier knowledge and norms.
    In such cases, the relevant quantity is when the text \emph{could have been written}, rather than when it actually was.
\end{enumerate}

